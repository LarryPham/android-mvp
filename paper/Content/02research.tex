

% An example of a floating figure using the graphicx package.
% Note that \label must occur AFTER (or within) \caption.
% For figures, \caption should occur after the \includegraphics.
% Note that IEEEtran v1.7 and later has special internal code that
% is designed to preserve the operation of \label within \caption
% even when the captionsoff option is in effect. However, because
% of issues like this, it may be the safest practice to put all your
% \label just after \caption rather than within \caption{}.
%
% Reminder: the "draftcls" or "draftclsnofoot", not "draft", class
% option should be used if it is desired that the figures are to be
% displayed while in draft mode.
%
%\begin{figure}[!t]
%\centering
%\includegraphics[width=2.5in]{myfigure}
% where an .eps filename suffix will be assumed under latex, 
% and a .pdf suffix will be assumed for pdflatex; or what has been declared
% via \DeclareGraphicsExtensions.
%\caption{Simulation Results}
%\label{fig_sim}
%\end{figure}

% Note that IEEE typically puts floats only at the top, even when this
% results in a large percentage of a column being occupied by floats.


% An example of a double column floating figure using two subfigures.
% (The subfig.sty package must be loaded for this to work.)
% The subfigure \label commands are set within each subfloat command, the
% \label for the overall figure must come after \caption.
% \hfil must be used as a separator to get equal spacing.
% The subfigure.sty package works much the same way, except \subfigure is
% used instead of \subfloat.
%
%\begin{figure*}[!t]
%\centerline{\subfloat[Case I]\includegraphics[width=2.5in]{subfigcase1}%
%\label{fig_first_case}}
%\hfil
%\subfloat[Case II]{\includegraphics[width=2.5in]{subfigcase2}%
%\label{fig_second_case}}}
%\caption{Simulation results}
%\label{fig_sim}
%\end{figure*}
%
% Note that often IEEE papers with subfigures do not employ subfigure
% captions (using the optional argument to \subfloat), but instead will
% reference/describe all of them (a), (b), etc., within the main caption.


% An example of a floating table. Note that, for IEEE style tables, the 
% \caption command should come BEFORE the table. Table text will default to
% \footnotesize as IEEE normally uses this smaller font for tables.
% The \label must come after \caption as always.
%
%\begin{table}[!t]
%% increase table row spacing, adjust to taste
%\renewcommand{\arraystretch}{1.3}
% if using array.sty, it might be a good idea to tweak the value of
% \extrarowheight as needed to properly center the text within the cells
%\caption{An Example of a Table}
%\label{table_example}
%\centering
%% Some packages, such as MDW tools, offer better commands for making tables
%% than the plain LaTeX2e tabular which is used here.
%\begin{tabular}{|c||c|}
%\hline
%One & Two\\
%\hline
%Three & Four\\
%\hline
%\end{tabular}
%\end{table}


% Note that IEEE does not put floats in the very first column - or typically
% anywhere on the first page for that matter. Also, in-text middle ("here")
% positioning is not used. Most IEEE journals use top floats exclusively.
% Note that, LaTeX2e, unlike IEEE journals, places footnotes above bottom
% floats. This can be corrected via the \fnbelowfloat command of the
% stfloats package.

\section{Research}

\IEEEPARstart{I}{n} this chapter, the questions are answered by research. Blablablabalabalaba.

\subsection{Components of an \emph{Android} Application}

Building an \emph{Android} application requires several core components of the \emph{Android} SDK to be used. This section gives a short overview and short explanation of the components, used for the prototype, created for this paper.

\textbf{Activity}: An application typically consists of one or more activities. An activity fills the whole screen of the device and can be compared to a \emph{window} in a classic desktop application. The official \emph{Android} SDK documentation \cite{SDKActivities} defines an activity as \enquote{an application component that provides a screen with which users can interact in order to do something}.
There is a defined lifcycle, which all activity objects go through -- the creation, start, stop and the destruction of an activity object. This lifecycle is managed by \emph{Android} itself, meaning an application developer must not manually create or destroy activity objects.

\textbf{Fragment}: Fragments split more complex parts of an activity's user interface into several smaller sub-components. Fragments are also used to encapsulate reusable parts of the user interface, which for example need to appear in mulitple activities. The official documentation of the \emph{Android} SDK defines a fragment as follows \cite{SDKFragments}:

\begin{quote}
A Fragment represents a behavior or a portion of user interface in an Activity. You can combine multiple fragments in a single activity to build a multi-pane UI and reuse a fragment in multiple activities.
\end{quote}


\subsection{Problems in Android Development (detailed)}

\subsection{MVP's two flavors: Passive View \& Supervising Controller}

\subsection{Approach to solving the problem}