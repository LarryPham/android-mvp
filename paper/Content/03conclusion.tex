\section{Conclusion}

\IEEEPARstart{I}{n} course of this research paper we have analyzed the characteristics given by the \emph{Android} SDK and have identified 
principle components which are generally used for developing \emph{Android} applications. Those components have been examined regarding their composition and characteristics. 
Finally we can offer an MVP approach for \emph{Android} and a corresponding prototype app demonstrating how principle \emph{Android} components can be structured for applying MVP.

However, just the essential SDK components have been examined regarding their MVP compatibility and used in the prototype. 
The MVP pattern for \emph{Android} would only succeed in practice if a wide bandwidth of SDK components was adaptable. 
In course of additional research studies addressing this topic we recommend to examine further SDK components and 
evaluate them regarding their MVP compatibility by implementing prototypes.

There are further and particularly similar MVP approaches developed and provided by the \emph{Android} developer community. 
\emph{Google} developer \emph{Chris Banes} has developed the \emph{Android} app \emph{philm} basing on MVP. 
The very demonstrative source code\footnote{\emph{philm} source code: \\\url{https://github.com/chrisbanes/philm}} is offered on \emph{github} and worth seeing. 
Moreover there is a huge developer community\footnote{\emph{Android MVP} community:\\\url{https://plus.google.com/communities/114285790907815804707}} on \emph{Google+}, dealing with different MVP approaches. 
Finally, we want to draw your attention to the paper \emph{Android Passive MVC: a Novel Architecture Model for Android Application Development} \cite{PassiveMVC}, which is worth reading and has been an inspiration for our approach. Although the paper proposes an architecture which is rather MVC than MVP-based, 
it offers an interesting approach how the components of an \emph{Android} app can be organized for building an MV-x structure. 