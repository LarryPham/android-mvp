\section{Introduction}

\IEEEPARstart{T}{he \emph{Android}} operating system is one of the most 
commonly used in mobile industry. After its first presentation in 2007, the open source system started its successful journey and became the leading mobile operating system in 2010 \cite{PassiveMVC}. According to Gartner \cite{GartnerAndroid2015}, there will be over 1.4 billion \emph{Android}-based devices sold in 2015.\\

What brings the actual value to an \emph{Android} device is the installed software, the so called \quotes{Applications} or 
\quotes{Apps}.
Software for \emph{Android} devices can be downloaded and installed via the 
\emph{Play Store}, a software distribution solution, developed by Google and 
pre-installed on every new device, which currently holds over 1.4 million 
applications (as of November 2014) \cite{AppBrainStats}.

% App Development

For developing such an application for the \emph{Android} mobile operating 
system, Google provides a software development kit (SDK) for the programming 
language \emph{Java}. It provides access to all functionality of the 
operating system and the underlaying phone- or tablet hardware. The principle components of the \emph{Android} SDK are explained later in this document.\\

% Challenges in Android Development

The SDK does not define a specific software architecture which has to 
be used for developing the application \cite{AndroidDeveloperCollection}. Instead, developers are totally free in 
designing their own architecture for fitting their needs for an application.
Having the number of 1.4 million applications in mind, this fact implies that 
there are various different software architectures currently implemented, each 
of them with their pros and cons. 

However, not having clear rules regarding the application's architecture entails risks, especially for people that are not expert-level developers. Applications become larger from time to time, and without a reasonable architecture this leads to complex code which is hard to understand and maintain \cite{PassiveMVC, BallOfMud}. Additionally, complex dependencies in the SDK cause messy code and make it hard to exchange specific elements without affecting other elements \cite{PassiveMVC}. That is especially because \emph{Android} tends to lead to classes containing a lot of functionality, which often requires more effort concerning testability and extendability \cite{GangOfFour}.

% Model-View-Presenter

A modern architecture approach is Model-View-Presenter (MVP), a model introduced in 1996 to support event-driven systems \cite{TaligentMVP}. Its main goal is to separate the Data Management (Model) and User Interface (View-Controller), thus creating software which is more flexible in terms of replacing the \emph{Model} without changing the presentation part or vice versa \cite{PassiveMVC}.
Potel's \emph{Model-View-Presenter} pattern consists of three components: \emph{Model}, \emph{View} and \emph{Presenter}. The \emph{Model} represents the domain model including the business logic. The \emph{View} represents the user interface and interacts with the user and displays the data. The \emph{Presenter} is the mediator between the business logic (Model) and the user interface (View). Its responsibility is to prepare the data (presentation logic), to handle user-initiated events and issues the appropriate manipulation of the data.

% Applying MVP in Android

Applying MVP to \emph{Android} applications seems promising to build less complex applications with clear code and improved extensibility. However, as the considerations in this paper show, an \emph{Android} implementation of MVP is tricky. Some components of the SDK have several responsibilities and make it hard to define whether they are a \emph{Model}, a \emph{View} or a \emph{Presenter}. 

The approach developed in this paper aims to clarify the distinction of these architectural parts. The findings of our research may serve as a guideline on how to apply MVP in \emph{Android} application development.

% Research Questions

In the following sections, these questions concerning the application of MVP to \emph{Android} application development are evaluated and answered:

\begin{enumerate}[label={RQ\arabic*:}, leftmargin=0.95cm]
\item What are the inherent characteristics given/recommended by the \emph{Android} framework and what is the current state in \emph{Android} application development in accordance to the inherent framework characteristics?
\item What are the key benefits of the \emph{Model-View-Presenter} approach?
\item How can the MVP approach be applied to the Android application architecture considering framework building blocks?
\end{enumerate}

To answer these questions, a prototypical app is built using the MVP approach. 

In the next section we shortly show what core components, provided by the \emph{Android} SDK, need to be considered for our MVP implementation. In addition, we elaborate what types of \emph{Model-View-Presenter} approaches exist and which one we have chosen to build our app prototype. This is followed by a detailed explanation how we applied MVP to the Android SDK and how we built our app prototype. The last section concludes our findings.
