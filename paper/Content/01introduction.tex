\section{Introduction}

\IEEEPARstart{T}{he \emph{Android}} operating system is one of the most 
commonly used in mobile industry. According to Gartner, there will be 1.1 
billion \emph{Android}-based devices sold in 2014 \cite{GartnerAndroid2014}.
But it's the installed software, the so called \quotes{Applications} or 
\quotes{Apps}, that bring the actual value to an \emph{Android} device.
Software for \emph{Android} devices can be downloaded and installed via the 
\emph{Play Store}, a software distribution solution, developed by Google and 
pre-installed on every new device, which currently holds over 1.4 million 
applications (as of November 2014) \cite{AppBrainStats}.

For developing such an application for the \emph{Android} mobile operating 
system, Google provides a software development kit (SDK) for the programming 
language \emph{Java}. It provides access to all functionality of the 
operating system and the underlaying phone- or tablet hardware.
However, the SDK does not define a concrete software architecture, which has to 
be used for developing the application. Instead, developers are totally free in 
designing their own architecture for fitting their needs for an application.
Having the number of 1.4 million applications in mind, this fact implies that 
there are various different software architectures currently implemented, each 
of them with their pros and cons.

\todo[inline]{
Vielleicht k\"onnen wir ja ein sch\"ones Paper damit erstellen und uns so schon einen Pluspunkt bei der Abgabe abholen. Ich bin mir sicher, dass der Prof. Zimmermann auf sch\"ones Aussehen steht.
}

	\subsection{Android}
	
	\subsection{MVC \& MVP}
	
	\subsection{Problems in Android Development}
	
	\subsection{Motivation: MVP for Android}
	
	\subsection{Research Questions}
	

	
	