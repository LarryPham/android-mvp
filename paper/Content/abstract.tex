\begin{abstract}
%\boldmath
Today, \emph{Android} is the most commonly used operating system in mobile industry. It allows a wide variety of architecture approaches in application development, all having their own advantages and disadvantages regarding testability, maintainability and the like. 
After introducing the basic components of \emph{Android} applications and giving an overview about the current state in \emph{Android} development, this paper describes how the Model-View-Presenter (MVP) approach can be used to develop applications that can be conveniently tested, maintained and extended.
To support the idea presented in this paper, an Android app called \emph{Yet Another Weather App} is created using the methods described in the paper.
\end{abstract}
% IEEEtran.cls defaults to using nonbold math in the Abstract.
% This preserves the distinction between vectors and scalars. However,
% if the journal you are submitting to favors bold math in the abstract,
% then you can use LaTeX's standard command \boldmath at the very start
% of the abstract to achieve this. Many IEEE journals frown on math
% in the abstract anyway.

% Note that keywords are not normally used for peerreview papers.
\begin{IEEEkeywords}
Android,  Application Architecture, Model-View-Presenter (MVP), App Prototype.
\end{IEEEkeywords}
